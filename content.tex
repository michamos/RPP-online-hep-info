\section{Introduction}\label{introduction}

The collection of online information resources in particle physics and
related areas presented in this chapter is of necessity incomplete. An
expanded and regularly updated online version can be found at:
\url{http://library.cern/particle_physics_information}

Suggestions for additions and updates are very welcome.\footnote{Please
  send comments and corrections to micha.moshe.moskovic@cern.ch}

\section{Particle Data Group (PDG)
resources}\label{particle-data-group-pdg-resources}

\begin{itemize}
\item
  \textbf{Review of Particle Physics (RPP):} A comprehensive report on
  the fields of particle physics and related areas of cosmology and
  astrophysics, including both review articles and a
  compilation/evaluation of data on particle properties. The review
  section includes articles, tables and plots on a wide variety of
  theoretical and experimental topics of interest to particle physicists
  and astrophysicists. The particle properties section provides tables
  of published measurements as well as the Particle Data Groups best
  values and limits for particle properties such as masses, widths,
  lifetimes, and branching fractions, as well as an extensive summary of
  searches for hypothetical particles. RPP is published as a large book
  every two years, with partial updates made available once each year on
  the web.

  All the contents of the book version of RPP are available online:
  \url{http://pdg.lbl.gov}

  The printed book can be ordered:
  \url{http://pdg.lbl.gov/2019/html/receive_our_products.html}

  Of historical interest is the complete RPP collection which can be
  found online: \url{http://pdg.lbl.gov/rpp-archive/}
  \url{http://library.cern/PDG_publications/review_particle_physics}
\item
  \textbf{Particle Physics booklet:} An abridged version of the Review
  of Particle Physics, available as a pocket-sized 250-page booklet. It
  is one of the most useful summaries of physics data. The booklet
  contains an abbreviated set of reviews and the summary tables from the
  most recent edition of the Review of Particle Physics.

  The PDF file of the booklet can be downloaded:
  \url{http://pdg.lbl.gov/current/booklet.pdf}

  The printed booklet can be ordered:
  \url{http://pdg.lbl.gov/2019/html/receive_our_products.html}
\item
  \textbf{PDGLive:} A web application for browsing the contents of the
  PDG database that contains the information published in the Review of
  Particle Physics. It allows one to navigate to a particle of interest,
  see a summary of the information available, and then proceed to the
  detailed information published in the Review of Particle Physics. Data
  entries are directly linked to the corresponding bibliographic
  information in INSPIRE. \url{http://pdglive.lbl.gov}
\item
  \textbf{Computer-readable files:} Data files that can be downloaded
  from PDG include tables of particle masses and widths, PDG Monte Carlo
  particle numbers, and cross-section data. The files are updated with
  each new edition of the Review of Particle Physics.
  \url{http://pdg.lbl.gov/current/html/computer_read.html}
\end{itemize}

\section{Particle Physics Information
Platforms}\label{particle-physics-information-platforms}

\begin{itemize}
\item
  \textbf{INSPIRE:} INSPIRE serves as a one-stop information platform
  for the particle physics community, comprising 8 interlinked databases
  on literature, conferences, institutions, journals, researchers,
  experiments, jobs and data. Run in collaboration by CERN, DESY,
  Fermilab, IHEP, IN2P3 and SLAC, it has been serving the scientific
  community for almost 50 years. Previously known as SPIRES, it was the
  first website outside Europe and the first database on the web. Close
  interaction with the user community and with arXiv, ADS, HepData,
  ORCID, PDG and publishers is the backbone of INSPIRE's evolution.
  \url{http://inspirehep.net/}

  In 2019, INSPIRE launched INSPIRE beta, featuring all new literature
  search, author profiles and job postings. INSPIRE beta is running in
  parallel with the current platform and it will fully replace it in the
  future. The INSPIRE beta site is available at:
  \url{http://beta.inspirehep.net}

  \begin{itemize}
  \tightlist
  \item
    blog:
    \href{http://blog.inspirehep.net/}{\texttt{http://blog.inspirehep.net/}}
  \item
    twitter: \href{https://twitter.com/inspirehep}{\texttt{@inspirehep}}
  \end{itemize}
\end{itemize}

\section{Literature Databases}\label{literature-databases}

\begin{itemize}
\item
  \textbf{ADS:} The SAO/NASA Astrophysics Data System is a Digital
  Library portal offering access to 13 million bibliographic records in
  Astronomy and Physics. The ADS search engine also indexes the
  full-text for approximately four million publications in this
  collection and tracks citations, which now amount to over 80 million
  links. The system also provides access and links to a wealth of
  external resources, including electronic articles hosted by publishers
  and arXiv, data catalogs and a variety of data products hosted by the
  astronomy archives worldwide. The ADS can be accessed at:
  \url{http://ads.harvard.edu/}
\item
  \textbf{arXiv.org:} A repository of full-text articles in physics,
  astronomy, mathematics, computer science, statistics, nonlinear
  sciences, quantitative finance, quantitative biology, electrical
  engineering and systems science, and economics. Papers are submitted
  by registered authors to arXiv, often as pre-prints in advance of
  submission to a journal for publication; includes post-prints, working
  papers, and other relevant material. Established in 1991, the
  repository is interlinked with ADS and INSPIRE, among others. Readers
  can browse subject categories or search by author, title, abstract,
  date, and other fields. Receive daily update alerts for subfields by
  email or RSS. \url{https://arXiv.org}

  \begin{itemize}
  \tightlist
  \item
    blog:
    \href{https://blogs.cornell.edu/arXiv}{\texttt{https://blogs.cornell.edu/arXiv}}
  \item
    twitter: \href{https://twitter.com/arxiv}{\texttt{@arxiv}}
  \end{itemize}
\item
  \textbf{CDS:} The CERN Document Server contains records of about
  700,000 CERN and non-CERN articles, preprints, theses. It includes
  records for internal and technical notes, official CERN committee
  documents, and multimedia objects. CDS is planning to focus on its
  role as an institutional repository covering all CERN material from
  the early 50s and reflecting the holdings of the CERN library.
  Non-CERN particle and accelerator physics content is in the process of
  being exported to INSPIRE. \url{http://cds.cern.ch}
\item
  \textbf{INSPIRE HEP:} The HEP collection, the flagship of the INSPIRE
  suite, serves more than 1.3 million bibliographic records with a
  growing number of fulltexts attached and metadata including author
  affiliations, abstracts, references, experiments, keywords as well as
  links to arXiv, ADS, PDG, HEPData, publisher platforms and other
  servers. It provides fast metadata and fulltext searches, plots
  extracted from fulltext, author disambiguation, author profile pages
  and citation analysis and is expanding its content to, e.g.,
  experimental notes. \url{http://inspirehep.net}
\item
  \textbf{JACoW:} The Joint Accelerator Conference Website publishes the
  proceedings of APAC, EPAC, PAC, IPAC, ABDW, BIW, COOL, CYCLOTRONS,
  DIPAC, ECRIS, FEL, HIAT, ICALEPCS, IBIC, ICAP, LINAC, North American
  PAC, PCaPAC, RuPAC, SRF. A custom interface allows searching based on
  keywords, titles, authors, and in the fulltext.
  \url{http://www.jacow.org/}
\item
  \textbf{KEK Library Preprints and Reports Database:} This database
  contains bibliographic records of preprints and technical reports held
  in the KEK library, with links to the full-text images of more than
  100,000 papers scanned from their worldwide preprint collection.
  Particularly useful for older scanned preprints. Links to it are
  included in INSPIRE HEP.
  \url{https://www.i-repository.net/il/meta_pub/engG0000128Lib}
\item
  \textbf{MathSciNet:} This database of almost 3 million items provides
  reviews, abstracts and bibliographic information for much of the
  mathematical sciences literature. Over 100,000 new items, most of them
  classified according to the Mathematics Subject Classification, and
  more than 80,000 reviews of the current published literature are added
  each year. Author identification allows users to search for
  publications per author and citation data allows to track the history
  and influence of research publications.
  \url{http://www.ams.org/mathscinet}
\item
  \textbf{OSTI.GOV:} A portal to free, publicly available DOE-sponsored
  R\&D results including technical reports, bibliographic citations,
  journal articles, conference papers, books, multimedia and data
  information. It consolidates OSTI's home page and the now retired
  primary search tool SciTech Connect. It contains over 3 million
  records, including citations to 1.5 million journal articles, 1
  million of which have digital object identifiers (DOIs) linking to
  full-text articles on publishers' websites. \url{https://www.osti.gov}
\end{itemize}

\section{Particle Physics Journals and Conference Proceedings
Series}\label{particle-physics-journals-and-conference-proceedings-series}

\begin{itemize}
\item
  \textbf{CERN Journal List:} This list of journals and conference
  series publishing particle physics content provides information on
  Open Access, copyright policies and terms of use.
  \url{http://library.cern/oa/where-publish}
\item
  \textbf{INSPIRE Journals:} The database contains over 3,600 journals
  publishing HEP-related articles.
  \url{http://inspirehep.net/collection/journals}
\end{itemize}

\section{Conference Databases}\label{conference-databases}

\begin{itemize}
\tightlist
\item
  \textbf{INSPIRE Conferences:} The database of more than 23,000 past,
  present and future conferences, schools, and meetings relevant to
  high-energy physics and related fields is searchable by title,
  acronym, series, date and location. Included are information about
  published proceedings, links to conference contributions in the
  INSPIRE HEP database, and links to the conference website when
  available. New conferences can be submitted from the entry page.
  \url{http://inspirehep.net/conferences}
\end{itemize}

\section{Research Institutions}\label{research-institutions}

\begin{itemize}
\tightlist
\item
  \textbf{INSPIRE Institutions:} The database of more than 11,500
  institutes, laboratories, and university departments in which research
  on particle physics and astrophysics is performed covers six
  continents and over a hundred countries. Included are address and web
  links where available as well as links to the papers from each
  institution in the HEP database, to scientists listed in HEPNames
  affiliated to this institution in the past or present and to
  experiments performed at this institution. Searches can be performed
  by name, acronym, location, etc. The site offers an alphabetical list
  by country as well as a list of the top 500 HEP and astrophysics
  institutions sorted by country.
  \url{http://inspirehep.net/institutions}
\end{itemize}

\section{People}\label{people}

\begin{itemize}
\tightlist
\item
  \textbf{INSPIRE HEPNames:} Searchable worldwide database of over
  125,000 active, retired and deceased people associated with particle
  physics and related fields. The affiliation history of these
  researchers, their e-mail addresses, ORCiDs, web pages, experiments
  they participated in, PhD advisor, information on their graduate
  students and links to their papers in the INSPIRE HEP, arXiv and ADS
  databases are provided as well as a user interface to update these
  informations. \url{http://inspirehep.net/hepnames}
\end{itemize}

\section{Experiments}\label{experiments}

\begin{itemize}
\item
  \textbf{INSPIRE Experiments:} Contains more than 3,500 past, present,
  and future experiments in particle physics. Lists both accelerator and
  non-accelerator experiments. Includes official experiment name and
  number, location, and collaboration lists. Simple searches by
  participant, title, experiment number, institution, date approved,
  accelerator, or detector, return a description of the experiment,
  including a complete list of authors, title, overview of the
  experiment's goals and methods, and a link to the experiment's web
  page if available. Recently, it has expanded its scope to include also
  particle accelerators besides experiments and link them together.
  \url{http://inspirehep.net/Experiments}
\item
  \textbf{Cosmic ray/Gamma ray/Neutrino and similar experiments:} This
  extensive collection of experiment websites is organized by focus of
  study and by location. Additional sections link to educational
  materials, organizations, and other useful resources. The site is
  maintained at the Max Planck Institute for Nuclear Physics,
  Heidelberg.
  \url{http://www.mpi-hd.mpg.de/hfm/CosmicRay/CosmicRaySites.html}
\end{itemize}

\section{Jobs}\label{jobs}

\begin{itemize}
\item
  \textbf{AAS Job Register:} The American Astronomical Society publishes
  once a month graduate, postgraduate, faculty and other positions
  mainly in astronomy and astrophysics.
  \url{http://jobregister.aas.org/}
\item
  \textbf{Academic Jobs Online}: A full-service online recruiting site
  for academic institutions worldwide in all disciplines and areas.
  \url{https://academicjobsonline.org/ajo}
\item
  \textbf{APS Careers:} A gateway for physicists, students, and physics
  enthusiasts to information about physics jobs and careers. It contains
  Physics job listings, career advice, upcoming workshops and meetings,
  and career and job related resources provided by the American Physical
  Society. \url{http://www.aps.org/careers/employment}
\item
  \textbf{brightrecruits.com:} A recruitment service run by IOP
  Publishing that connects employers from different industry sectors
  with jobseekers who have a background in physics and engineering.
  \url{http://brightrecruits.com/}
\item
  \textbf{IOP Careers:} Career information and resources primarily aimed
  at university students are provided by the UK Institute of Physics.
  \url{http://www.iop.org/careers/}
\item
  \textbf{INSPIRE HEPJobs:} Lists academic and research jobs in high
  energy physics, nuclear physics, accelerator physics and astrophysics
  with the option to post a job or to receive email notices of new job
  listings. About 500 jobs are currently listed.
  \url{http://inspirehep.net/jobs}
\item
  \textbf{Physics Today Jobs:} Online recruitment advertising website
  for Physics Today magazine, published by the American Institute of
  Physics. Physics Today Jobs is the managing partner of the AIP Career
  Network, an online job board network for the physical science,
  engineering, and computing disciplines. 8,000 resumes are currently
  available, and more than 2,500 jobs were posted in 2012.
  \url{http://www.physicstoday.org/jobs}
\end{itemize}

\section{Software Packages and
Repositories}\label{software-packages-and-repositories}

\subsection{Repositories}\label{repositories}

\begin{itemize}
\item
  \textbf{ASCL:} The Astrophysics Source Code Library (ASCL) is a free
  online registry for source codes of interest to astronomers and
  astrophysicists and lists codes that have been used in research that
  has appeared in, or been submitted to, peer-reviewed publications.
  \url{http://ascl.net}
\item
  \textbf{FreeHEP:} A collection of software and information about
  software useful in high-energy physics and adjacent disciplines,
  focusing on open-source software for data analysis and visualization.
  Searching can be done by title, subject, date acquired, date updated,
  or by browsing an alphabetical list of all packages.
  \url{http://java.freehep.org/}
\item
  \textbf{GenSer:} The Generator Services project collaborates with
  Monte Carlo (MC) generators authors and with LHC experiments in order
  to prepare validated LCG compliant code for both theoretical and
  experimental communities at the LHC, sharing the user support duties,
  providing assistance for the development of the new object-oriented
  generators and guaranteeing the maintenance of the older packages on
  the LCG supported platforms. The project consists of the generators
  repository, validation, HepMC record and MCDB event databases.
  \url{http://ep-dep-sft.web.cern.ch/project/generator-service-project-genser}
\item
  \textbf{Hepforge:} A development environment for high-energy physics
  software projects, in particular housing many event-generator related
  projects that offers a ready-made, easy-to-use set of web-based tools,
  including shell account with up-to-date development tools, web page
  hosting, subversion, git and Mercurial code management systems,
  mailing lists, bug tracker and wiki system.
  \url{http://www.hepforge.org/}
\end{itemize}

\subsection{Particle Physics software}\label{particle-physics-software}

General purpose software packages:

\begin{itemize}
\item
  \textbf{FastJet:} This is a software package for jet finding in pp and
  e+e- collisions. It includes fast native implementations of many
  sequential recombination clustering algorithms, plugins for access to
  a range of cone jet finders and tools for advanced jet manipulation.
  \url{http://fastjet.fr/}
\item
  \textbf{GAMBIT:} A global fitting code for generic Beyond the Standard
  Model theories, designed to allow fast and easy definition of new
  models, observables, likelihoods, scanners and backend physics codes.
  \url{http://gambit.hepforge.org}
\item
  \textbf{Geant4:} This is a toolkit for the simulation of the passage
  of particles through matter. Its areas of application include high
  energy, nuclear and accelerator physics, as well as studies in medical
  and space science. \url{http://geant4.web.cern.ch/geant4/}
\item
  \textbf{LHAPDF:} HEP community standard library for parton
  distribution function evolution and querying, including official
  collection of PDF data sets. \url{http://lhapdf.hepforge.org/}
\item
  \textbf{QUDA:} Library for performing calculations in lattice QCD on
  GPUs using NVIDIA's CUDA platform. The current release includes
  optimized solvers for Wilson, Clover-improved Wilson,Twisted mass,
  Staggered, Improved staggered, Domain wall and Mobius fermion actions.
  \url{http://lattice.github.io/quda/}
\item
  \textbf{Rivet:} The Rivet toolkit, a system for validation of Monte
  Carlo event generators, provides a large set of experimental analyses
  useful for MC generator development, validation, and tuning.
  \url{http://rivet.hepforge.org/}
\item
  \textbf{ROOT:} This framework for data processing in high-energy
  physics, born at CERN, offers applications to store, access, process,
  analyze and represent data or perform simulations.
  \url{http://root.cern.ch}
\item
  \textbf{Scikit-HEP:} This is a community-driven and community-oriented
  project with the aim of providing Particle Physics at large with an
  ecosystem for data analysis in Python. The project started in Autumn
  2016 and is under active development. It focuses on providing core and
  common tools for the community but also on improving the
  interoperability between HEP tools and the scientific ecosystem in
  Python as well as the discoverability of utility packages and
  projects.
\item
  \textbf{TMDplotter:} Allows to plot TMDs and PDFs as a function of
  different variables. \url{http://tmdplotter.desy.de/}
\item
  \textbf{tmLQCD:} This freely available software suite provides a set
  of tools to be used in lattice QCD simulations, mainly a HMC
  implementation for Wilson and Wilson twisted mass fermions and
  inverter for different versions of the Dirac operator.
  \url{https://github.com/etmc/tmLQCD}
\item
  \textbf{USQCD:} The software suite enables lattice QCD computations to
  be performed with high performance across a variety of architectures.
  The page contains links to the project web pages of the individual
  software modules, as well as to complete lattice QCD application
  packages which use them. \url{http://usqcd-software.github.io}
\item
  \textbf{Software lists:} A list of Monte Carlo generators may be found
  at:
  \url{http://cmsdoc.cern.ch/cms/PRS/gentools/www/geners/collection/}

  The homepage of the SUSY Les Houches Accord contains links to codes
  relevant for supersymmetry calculations and phenomenology.
  \url{http://skands.physics.monash.edu/slha/}

  A variety of codes and algorithmic tools for analysing supersymmetric
  phenomenology is described in \url{http://arxiv.org/abs/0805.2088}

  G. Cowan's list provides links to HEP software, general statistics and
  data analysis links.
  \url{http://www.pp.rhul.ac.uk/~cowan/sda/statlinks.html}

  An extended list of more specialized HEP-related software can be found
  in the online version of this review:
  \url{http://library.cern/particle_physics_information\#sof}
\end{itemize}

\subsection{Astrophysics Software}\label{astrophysics-software}

\begin{itemize}
\item
  \textbf{Astropy:} The Astropy Project is a community effort to develop
  a single core package for Astronomy in Python and foster
  interoperability between Python astronomy packages.
  \url{http://www.astropy.org}
\item
  \textbf{IRAF:} The Image Reduction and Analysis Facility is a general
  purpose software system for the reduction and analysis of astronomical
  data. IRAF is written and supported by the National Optical Astronomy
  Observatories (NOAO) in Tucson, Arizona. \url{http://iraf.noao.edu/}
\item
  \textbf{Starlink:} Starlink was a UK Project supporting astronomical
  data processing. It was shut down in 2005 but its open-source software
  continued to be developed at the Joint Astronomy Centre until March
  2015. It is currently maintained by the East Asian Observatory. The
  open-source software products are a collection of applications and
  libraries, usually focused on a specific aspect of data reduction or
  analysis. \url{http://starlink.eao.hawaii.edu/starlink}
\item
  Links to a large number of astronomy software archives are listed at:
  \url{http://heasarc.nasa.gov/docs/heasarc/astro-update/}
\end{itemize}

\subsection{Web Apps}\label{web-apps}

\begin{itemize}
\item
  \textbf{APFEL:} This online parton density function plotter allows to
  compare predictions for different PDF fits.
  \url{https://apfel.mi.infn.it/}
\item
  \textbf{ColliderReach:} A tool to give a simple estimate of the
  relation between the mass reaches of different proton-proton collider
  configurations. \url{http://collider-reach.web.cern.ch/}
\end{itemize}

\subsection{Mobile Apps}\label{mobile-apps}

\begin{itemize}
\item
  \textbf{arXiv browser:} Android app for browsing and searching
  arXiv.org, and for reading, saving and sharing
  articles.\url{https://play.google.com/store/apps/details?id=com.commonsware.android.arXiv}
\item
  \textbf{aNarXiv:} arXiv viewer.
  \url{http://github.com/nephoapp/anarxiv}
\item
  \textbf{Collider:} This mobile app allows to see data from the ATLAS
  experiment at the LHC. \url{http://collider.physics.ox.ac.uk/}
\item
  \textbf{LHSee:} This smartphone app allows to see collisions from the
  Large Hadron Collider.
  \url{http://www2.physics.ox.ac.uk/about-us/outreach/public/lhsee}
\item
  \textbf{The Particles:} App for Apple iPad, Windows 8 and Microsoft
  Surface. Allows to browse a wealth of real ``event'' images and
  videos, read popular ``biographies'' of each of the particles and
  explore the A-Z of particle physics with its details and definitions
  of key concepts, laboratories and physicists. Developed by Science
  Photo Library in partnership with Prof.~Frank Close.
  \url{http://www.sciencephoto.com/apps/particles.html}
\end{itemize}

\section{Data repositories}\label{data-repositories}

\subsection{Particle Physics}\label{particle-physics}

\begin{itemize}
\item
  \textbf{HEPData:} The HEPData project, funded by the STFC (UK) and
  based at Durham University, has been built up over the past four
  decades as a unique repository for scattering data from experimental
  particle physics papers. It currently comprises the data points from
  plots and tables related to several thousand publications including
  those from the LHC. The data from HEPData can also be accessed through
  INSPIRE. A new enhanced service was recently developed in
  collaboration with CERN. \url{https://hepdata.net}
\item
  \textbf{CERN Open Data:} The CERN Open Data portal provides data from
  real collision events, as well as simulated and simplified datasets,
  produced by the experiments at the LHC; virtual machines to reproduce
  the analysis environment and software to process them. It serves over
  2 PB of data in total and encourages their use for both educational
  and research purposes. \url{http://opendata.cern.ch}
\item
  \textbf{HepSim:} A repository with Monte Carlo simulations for
  particle-collision experiments. It contains predictions from parton
  shower models and includes Monte Carlo events after fast and full
  detector simulations and event reconstruction.
  \url{http://atlaswww.hep.anl.gov/hepsim/}
\item
  \textbf{ILDG:} The International Lattice Data Grid is an international
  organization which provides standards, services, methods and tools
  that facilitate the sharing and interchange of lattice QCD gauge
  configurations among scientific collaborations by uniting their
  regional data grids. It offers semantic access with local tools to
  worldwide distributed data. \url{http://www.usqcd.org/ildg/}
\item
  \textbf{MCDB - Monte Carlo Database:} This central database of MC
  events aims to facilitate communication between Monte-Carlo experts
  and users of event samples in LHC collaborations. Having these events
  stored in a public place along with the corresponding documentation
  allows for direct cross checks of the performances on reference
  samples. \url{http://mcdb.cern.ch/}
\item
  \textbf{MCPLOTS:} MCPLOTS is a repository of Monte Carlo plots
  comparing High Energy Physics event generators to a wide variety of
  available experimental data. The website is supported by the LHC
  Physics Centre at CERN. \url{http://mcplots.cern.ch/}
\end{itemize}

\subsection{Astrophysics}\label{astrophysics}

\begin{itemize}
\item
  \textbf{CfA Dataverse:} This astronomy data repository at Harvard is
  open to all scientific data from astronomical institutions worldwide.
  \url{https://dataverse.harvard.edu/dataverse/cfa}
\item
  \textbf{NASA's HEASARC:} The High Energy Astrophysics Science Archive
  Research Center (HEASARC) is the primary archive for NASA's (and other
  space agencies') missions dealing with electromagnetic radiation from
  extremely energetic phenomena ranging from black holes to the Big
  Bang. \url{http://heasarc.gsfc.nasa.gov/}
\item
  \textbf{NASA archives:} The NASA archives provide access to raw and
  processed datasets from numerous NASA missions.

  Mikulski Archive for Space Telescopes (MAST): Hubble telescope, other
  missions (UV, optical): \url{http://archive.stsci.edu/}

  NASA/IPAC Infrared Science Archive: Spitzer, Herschel, Planck
  telescope, other missions: \url{http://irsa.ipac.caltech.edu/}
\item
  \textbf{NASA/IPAC Extragalactic Database (NED):} An astronomical
  database that collates and cross-correlates information on
  extragalactic objects. It contains their positions, basic data, and
  names as well as bibliographic references to published papers, and
  notes from catalogs and other publications. NED supports searches for
  objects and references, and offers browsing capabilities for abstracts
  of articles of extragalactic interest.
  \url{http://ned.ipac.caltech.edu/}
\item
  \textbf{SIMBAD:} The SIMBAD astronomical database provides basic data,
  cross-identifications, bibliography and measurements for astronomical
  objects outside the solar system. It can be queried by object name,
  coordinates and various criteria. Lists of objects and scripts can be
  submitted. \url{http://simbad.u-strasbg.fr/simbad/}
\item
  \textbf{VizieR:} VizieR provides access to the most complete library
  of published astronomical catalogues and data tables, available online
  organized in a self-documented database. Query tools allow users to
  select relevant data tables and extract and format records matching
  given criteria. Currently, more than 19,000 catalogues are available.
  \url{http://vizier.u-strasbg.fr/}
\end{itemize}

\subsection{General Physics}\label{general-physics}

\begin{itemize}
\item
  \textbf{NIST Physical Measurement Laboratory:} The National Institute
  of Standards and Technology provides access to physical reference data
  (physical constants, atomic spectroscopy data, x-ray and gamma-ray
  data, radiation dosimetry data, nuclear physics data and more) and
  measurements and calibrations data (dimensional and electromagnetic
  measurements). \url{https://www.nist.gov/pml/}
\item
  \textbf{Springer Materials - The Landolt-Börnstein Database:}
  Landolt-Börnstein is a data collection covering all areas of physical
  sciences and engineering, such as particle physics, electronic
  structure and transport, magnetism, superconductivity. International
  experts scan the primary literature in more than 8,000 peer-reviewed
  journals and evaluate and select the most valid information to be
  included in the database. It includes more than 130,000 online
  documents, 1,2 million references, and covers 250,000 chemical
  substances. SPringerMaterials Interactive allows to visualise and
  analyse data. The search functionality is freely accessible and the
  search results are displayed in their context, whereas the full text
  is secured to subscribers. \url{http://materials.springer.com}
\end{itemize}

\section{Data preservation
activities}\label{data-preservation-activities}

\subsection{Particle Physics}\label{particle-physics-1}

\begin{itemize}
\item
  \textbf{CERN Analysis Preservation:} CERN Analysis Preservation is a
  platform for preserving knowledge and assets of individual physics
  analyses in LHC collaborations. Its aim is to capture and document all
  the elements needed to understand and rerun an analysis even several
  years later: data, software, environment, workflow, context, and
  documentation. This platform is currently in a pilot stage. It is
  accessible by LHC experimental groups (standard collaboration access
  restrictions are applied). \url{https://analysispreservation.cern.ch}
\item
  \textbf{DASPOS:} A collective effort to explore the realisation of a
  viable data, software and algorithm preservation architecture in High
  Energy Physics \url{https://daspos.crc.nd.edu}
\item
  \textbf{DPHEP:} DPHEP coordinates the efforts to define and implement
  Data Preservation and Long Term Analysis in HEP. DPHEP, which was
  initiated as a study group in 2008-2009, includes all major HEP
  experiments and labs. In 2014, it has become a Collaboration through
  the signature of a Collaboration Agreement by a number of large
  funding agencies. The group is endorsed by the International Committee
  for Future Accelerators (ICFA).

  DPHEP regularly organizes workshops, creates status reports, and
  maintains links with similar activities in other disciplines. Details
  of the organizational structure, the objectives, workshops and
  publications can be found on the website. \url{http://dphep.org}
\item
  \textbf{REANA:} REANA (REusable ANAlyses) is a system for
  instantiating research data analyses on the cloud using
  container-based solutions. It complements CERN Analysis Preservation
  permitting the reuse and revalidation of preserved analyses. It is
  being developed in close collaboration with DASPOS and RECAST.
  \url{http://reanahub.io/}
\item
  \textbf{RECAST:} Building on analysis preservation and re-use
  infrastructure of the LHC experiments, RECAST acts as a science
  gateway allowing theorists to suggest new reinterpretations of
  archived analyses of the LHC dataset. Experiments review suggestions
  and if approved, simulate the proposed models and re-run the archived
  analysis to determine their viability. Such reinterpretation results
  are then appended to the records of the original publication in the
  relevant digital archives. Its new website should be available soon
  at: \url{http://recast.cern.ch}
\end{itemize}

\subsection{Astrophysics}\label{astrophysics-1}

More formal and advanced data preservation activity is ongoing in the
field of Experimental Astrophysics, including:

\begin{itemize}
\tightlist
\item
  Fermi Data \url{https://fermi.gsfc.nasa.gov/ssc/data}
\item
  IVOA (International Virtual Observatory Alliance)
  \url{http://www.ivoa.net/astronomers/applications.html}
\item
  LOSC (LIGO Open Science Center) \url{https://losc.ligo.org/about/}
\item
  PLA (Planck Legacy Archive) \url{http://pla.esac.esa.int/pla/}
\item
  SDSS (Sloan Digital Sky Survey) \url{http://sdss.org}
\end{itemize}

\section{Particle Physics Education and Outreach
Sites}\label{particle-physics-education-and-outreach-sites}

A useful list of resources can also be found at
\url{http://www.stfc.ac.uk/research/particle-physics-and-particle-astrophysics/particle-physics-resources/}

\subsection{Science Educators'
Networks}\label{science-educators-networks}

\begin{itemize}
\item
  \textbf{IPPOG:} The International Particle Physics Outreach Group is a
  network of particle physicists, researchers, informal science
  educators and science explainers aiming to raise awareness,
  understanding and standards of global outreach efforts in particle
  physics and general science by providing discussion forums and regular
  information exchange for science institutions, proposing and
  implementing strategies to share lessons learned and best practices
  and promoting current outreach efforts of network members.
  \url{http://ippog.web.cern.ch}
\item
  \textbf{Interactions.org:} Designed to serve as a central resource for
  communicators of particle physics. The daily updated website provides
  links to current particle physics news from the world's press,
  high-resolution photos and graphics from the particle physics
  laboratories of the world; links to education and outreach programs;
  information about science policy and funding; a glossary; and links to
  many educational sites. \url{http://www.interactions.org}
\item
  \textbf{I2U2 (Interactions in Understanding the Universe):} The I2U2
  e-Labs use the Internet and distributed computing in high-school
  classes and provide an opportunity for students to organise and
  conduct authentic research; experience the environment of scientific
  collaborations; make real scientific contributions. It is supported by
  QuarkNet, NSF and DOE. \url{http://www.i2u2.org}
\end{itemize}

\textbf{Physics Courses}

\begin{itemize}
\item
  \textbf{MIT OpenCourseWare - Physics:} These MIT course materials
  reflect almost all the undergraduate and graduate subjects taught at
  MIT. In addition to physics courses, supplementary educational
  resources are also available.
  \url{http://ocw.mit.edu/courses/physics/}
\item
  \textbf{OnlineCourses.com:} A collection of online tests, video
  lectures, and related course materials from mostly prestigious
  universities around the world.
  \url{http://www.onlinecourses.com/physics/}
\end{itemize}

\subsection{Master Classes}\label{master-classes}

\begin{itemize}
\item
  \textbf{International Masterclasses:} Each year about 10000 high
  school students in 42 countries come to one of about 200 nearby
  universities or research centres for one day in order to unravel the
  mysteries of particle physics. Lectures from active scientists give
  insight in topics and methods of basic research at the fundaments of
  matter and forces, enabling the students to perform measurements on
  real data from particle physics experiments themselves. At the end of
  each day, like in an international research collaboration, the
  participants join in a video conference for discussion and combination
  of their results. \url{http://physicsmasterclasses.org/}
\item
  \textbf{LHC physics masterclasses:} Lectures from active scientists
  give insight into methods of basic research, enabling the students to
  perform measurements on real data from LHC experiments. Like in a real
  research collaboration, the participants then discuss their results
  and compare with expectations.

  CMS: \url{http://cms.web.cern.ch/content/cms-physics-masterclass}

  \url{http://lhcb-public.web.cern.ch/lhcb-public/en/LHCb-outreach/masterclasses/en}

  \url{http://alice.physicsmasterclasses.org/MasterClassWebpage.html}
\item
  \textbf{IceCube:} \url{https://masterclass.icecube.wisc.edu/}
\item
  \textbf{MINERVA:} MINERVA (Masterclass INvolving Event recognition
  visualised with Atlantis) is a masterclass tool for students looking
  to learn more about the ATLAS experiment at CERN, based on a
  simplified setup of the ATLAS event display, Atlantis.
  \url{http://atlas-minerva.web.cern.ch/atlas-minerva/}
\end{itemize}

\subsection{General Sites}\label{general-sites}

\begin{itemize}
\item
  \textbf{Contemporary Physics Education Project (CPEP):} Provides
  charts, brochures, Web links, and classroom activities. Online
  interactive courses include: Fundamental Particles and Interactions;
  Plasma Physics and Fusion; History and Fate of the Universe; and
  Nuclear Science. \url{http://www.cpepweb.org/}
\item
  \textbf{PhysicsCentral:} This site maintained by the American Physical
  Society provides information about current research and people in
  physics, experiments that can be performed at home or at school and
  the possibility to get physics questions answered by physicists.
  \url{http://www.physicscentral.com}
\end{itemize}

\textbf{General Physics Activities}

\begin{itemize}
\item
  \textbf{HyperPhysics:} An exploration environment for concepts in
  physics employing concept maps and other linking strategies and
  providing opportunities for numerical exploration.
  \url{http://hyperphysics.phy-astr.gsu.edu/hbase/hph.html}
\item
  \textbf{Physics2000:} An interactive journey through modern physics.
  Have fun learning visually and conceptually about 20th century science
  and high-tech devices. Supported by the Colorado Commission on Higher
  Education and the National Science Foundation.
  \url{http://www.colorado.edu/physics/2000}
\end{itemize}

\subsection{Particle Physics
Activities}\label{particle-physics-activities}

\begin{itemize}
\item
  \textbf{Angels and Demons:} With the aim of looking at the myth versus
  reality of antimatter and science at CERN this site offers teacher
  resources, slideshows and videos of talks given to teachers visiting
  CERN. \url{http://angelsanddemons.web.cern.ch/}
\item
  \textbf{Big Bang Science: Exploring the origins of matter:} This
  website, produced by the Particle Physics and Astronomy Research
  Council of the UK (PPARC), explains what physicists are looking for
  with their giant instruments. It focuses on CERN particle detectors
  and on United Kingdom scientists' contribution to the search for the
  fundamental building blocks of matter.
  \url{http://hepwww.rl.ac.uk/pub/bigbang/part1.html}
\item
  \textbf{Cambridge Relativity and Cosmology:}
  \url{http://www.damtp.cam.ac.uk/research/gr/public/index.html}
\item
  \textbf{CAMELIA:} CAMELIA (Cross-platform Atlas Multimedia Educational
  Lab for Interactive Analysis) is a discovery tool for the general
  public, based on computer gaming technology.
  \url{https://www.atlasexperiment.org/camelia.html}
\item
  \textbf{CERNland:} With a range of games, multimedia applications and
  films CERNland is a virtual theme park developed to bring the
  excitement of CERN's research to a young audience aged between 7 and
  12. CERNland is designed to show children what is being done at CERN
  and inspire them with some physics at the same time.
  \url{http://www.cernland.net/}
\item
  \textbf{CollidingParticles:} A series of films following a team of
  physicists involved in research at the LHC.
  \url{http://www.collidingparticles.com/}
\item
  \textbf{Hands-On Universe:} This educational program enables students
  to investigate the Universe while applying tools and concepts from
  science, math and technology. \url{http://handsonuniverse.org/}
\item
  \textbf{Higgs Hunters:} A web-based citizen science project to help
  search for unknown exotic particles in the LHC data.
  \url{http://HiggsHunters.org}
\item
  \textbf{HYPATIA:} HYPATIA (Hybrid Pupil's Analysis Tool for
  Interactions in Atlas) is a tool for high school students to inspect
  the graphic visualization of particle collision products in the ATLAS
  detector at CERN. \url{http://hypatia.phys.uoa.gr/}
\item
  \textbf{Imagine the Universe:} This NASA site is intended for students
  age 14 and up and for anyone interested in learning about the
  universe. \url{http://imagine.gsfc.nasa.gov/home.html}
\item
  \textbf{In particular:} Podcast about physics and the process of
  discovering physics at the ATLAS experiment.
  \url{https://inparticular.web.cern.ch/}
\item
  \textbf{Lancaster Particle Physics:} This site, suitable for 16+
  students, offers a number of simulations and explanations of particle
  physics, including a section on the LHC.
  \url{http://www.lppp.lancs.ac.uk/}
\item
  \textbf{LHC @ home:} Volunteer computing platform to help physicists
  compare theory with experiment, in the search for new fundamental
  particles and answers to questions about the Universe.
  \url{http://lhcathome.web.cern.ch}

  ATLAS @ Home is a research project that uses volunteer computing to
  run simulations of the ATLAS experiment at CERN.
  \url{http://lhcathome.web.cern.ch/projects/atlas}

  Beauty allows volunteers to participate in simulations of the LHCb
  experiment at CERN. \url{http://lhcathome.web.cern.ch/projects/beauty}

  CMS @ Home is currently under development.
  \url{http://lhcathome.web.cern.ch/projects/cms}

  The SIXTRACK project allows users with Internet-connected computers to
  participate in advancing Accelerator Physics.
  \url{http://lhcathome.web.cern.ch/projects/sixtrack}

  Test4Theory allows volunteers to run simulations of high-energy
  particle physics on their home computers. The results are submitted to
  a database which is used as a common resource by both experimental and
  theoretical scientists working on the Large Hadron Collider at CERN.
  \url{http://lhcathome.web.cern.ch/projects/test4theory}
\item
  \textbf{Particle Adventure:} One of the most popular websites for
  learning the fundamentals of matter and force. An award-winning
  interactive tour of quarks, neutrinos, antimatter, extra dimensions,
  dark matter, accelerators and particle detectors from the Particle
  Data Group of Lawrence Berkeley National Laboratory. Simple elegant
  graphics and translations into 16 languages.
  \url{http://particleadventure.org/}
\item
  \textbf{Phantom of the Universe:} A planetarium show about dark matter
  that covers astrophysics, an underground experiment, and the LHC. The
  show is present in more than 510 planetariums in 66 countries
  worldwide and has been translated into 21 languages. It has been seen
  by about 4 million people. It is distributed to planetariums for free.
  \url{http://phantomoftheuniverse.com/}
\item
  \textbf{Quarked! - Adventures in the Subatomic Universe:} This
  project, targeted to kids aged 7-12 (and their families), brings
  subatomic physics to life through a multimedia project including an
  interactive website, a facilitated program for museums and schools,
  and an educational outreach program. \url{http://www.quarked.org/}
\item
  \textbf{QuarkNet:} Brings the excitement of particle physics research
  to high school teachers and their students. Teachers join research
  groups at about 50 universities and labs across the country. These
  research groups are part of particle physics experiments at CERN or
  Fermilab. About 100,000 students from 500+ US high schools learn
  fundamental physics as they participate in inquiry-oriented
  investigations and analyze real data online. QuarkNet is supported in
  part by the National Science Foundation and the U.S. Department of
  Energy. \url{https://quarknet.org/}
\item
  \textbf{Rewarding Learning videos about CERN:} The three videos based
  on interviews with scientists and engineers at CERN introduce pupils
  to CERN and the type of research and work undertaken there and are
  accompanied by teachers' notes.
  \url{http://www.nicurriculum.org.uk/STEMWorks/resources/cern/index.asp}
\end{itemize}

\subsection{Lab Education Offices}\label{lab-education-offices}

\begin{itemize}
\item
  \textbf{Argonne National Laboratory (ANL) Educational Programs:}
  \url{http://www.anl.gov/education/}
\item
  \textbf{Brookhaven National Laboratory (BNL) Educational Programs:}
  The Office of Educational Programs mission is to design, develop,
  implement, and facilitate workforce development and education
  initiatives that support the scientific mission at Brookhaven National
  Laboratory and the Department of Energy.
  \url{http://www.bnl.gov/education/}
\item
  \textbf{CERN:} The CERN Teacher Programmes help teachers keep up to
  date with the latest developments in particle physics and related
  areas and enables to meet teaching colleagues from around the world.
  \url{http://teacher-programmes.web.cern.ch/}

  In the High-School Students Internship Programme CERN invites students
  aged 16-19 to come to CERN for two weeks, to gain practical experience
  in science, technology, and innovation by shadowing, observing, and
  working with a member of personnel. \url{http://hssip.web.cern.ch/}
\item
  \textbf{DESY:} Offers courses for pupils and teachers as well as
  information for the general public, mostly in German.
  \url{http://www.desy.de/information__services/education/}
\item
  \textbf{Fermilab Education Office:} Provides education resources and
  information about activities for educators, physicists, students and
  visitors to the Lab. In addition to information about 25 programs, the
  website provides online data-based investigations for high school
  students, online versions of exhibits in the Lederman Science Center,
  links to particle physics discovery resources, web-based instructional
  resources, tips for education and outreach, and links to the Lederman
  Science Center and the Teacher Resource Center.
  \url{http://ed.fnal.gov/}
\item
  \textbf{Science Education at Jefferson Lab:}
  \url{http://education.jlab.org/}
\item
  \textbf{LBL Workforce Development and Education:} This group carries
  out Berkeley Lab's mission to inspire and prepare the next generation
  of scientists, engineers, and technicians.
  \url{https://education.lbl.gov/}
\end{itemize}

\subsection{Educational Programs of
Experiments}\label{educational-programs-of-experiments}

\begin{itemize}
\item
  \textbf{ATLAS Discovery Quest:} One of several access points to ATLAS
  education and outreach pages. This page gives access to explanations
  of physical concepts, blogs, ATLAS facts, news, and information for
  students and teachers. \url{http://www.atlas.ch/physics.html}
\item
  \textbf{ATLAS eTours:} Gives a description of the Large Hadron
  Collider, explain how the ATLAS detector at the LHC works and give an
  overview over the experiments and their physics goals.
  \url{http://atlasexperiment.org/etours.html}
\item
  \textbf{CMS Education:} Provides access to educational resources
  (Story of the Universe, The Size of Things, What is a Particle), and
  to multimedia material, such as interviews, movies and photos.
  \url{http://cms.web.cern.ch/tags/education}
\item
  \textbf{Education and Outreach @ IceCube:}
  \url{http://icecube.wisc.edu/outreach}
\item
  \textbf{LIGO Science Education Center:} The LIGO (Laser Interferometer
  Gravitational-wave Observatory) Science Education Center has over 40
  interactive, hands-on exhibits that relate to the science of LIGO. The
  website hosts field trips for students, teacher training programs, and
  tours for the general public. Visitors can explore scientific concepts
  such as light, gravity, waves, and interference; learn about LIGO's
  search for gravitational waves; and interact with scientists and
  engineers. \url{https://ligo.caltech.edu/page/educational-resources}
\item
  \textbf{Pierre Auger Observatory's Educational Pages:} The site offers
  information about cosmic rays and their detection, and provides
  material for students and teachers.
  \url{https://www.auger.org/index.php/edu-outreach}
\end{itemize}

\subsection{News}\label{news}

\begin{itemize}
\item
  \textbf{Asimmetrie:} Bimonthly magazine about particle physics
  published by INFN, the Istituto Nazionale di Fisica Nucleare
  \url{http://www.asimmetrie.it/}
\item
  \textbf{CERN Courier:} \url{https://cerncourier.com}
\item
  \textbf{DESY inForm:} \url{http://www.desy.de/aktuelles/desy_inform}
\item
  \textbf{Fermilab news:} \url{https://news.fnal.gov}
\item
  \textbf{LC Newsline:} The newsletter of the Linear Collider community.
  \url{http://newsline.linearcollider.org/}

  \begin{itemize}
  \tightlist
  \item
    twitter:
    \href{https://twitter.com/ILCnewsline}{\texttt{@ILCnewsline}}
  \end{itemize}
\item
  \textbf{IOP News:} \url{http://www.iop.org/news/}
\item
  \textbf{JINR News:} \url{http://www1.jinr.ru/News/Jinrnews_index.html}
\item
  \textbf{News at Interactions.org:} The InterActions site provides news
  and press releases on particle physics.
  \url{http://www.interactions.org/news-center}

  \begin{itemize}
  \tightlist
  \item
    twitter:
    \href{https://twitter.com/particlenews}{\texttt{@particlenews}}
  \end{itemize}
\item
  \textbf{physics.org news:} This IOP news site presents physics stories
  from around the world wide web. \url{http://www.physics.org/news.asp}
\item
  \textbf{SLAC Signals:} This email newsletter reports about
  cutting-edge science, major SLAC milestones and other lab information.
  It has replaced SLAC Today in November 2013. Its signup page can be
  found at \url{http://eepurl.com/IqPl1}
\item
  \textbf{Symmetry:} This magazine about particle physics and its
  connections to other aspects of life and science, from
  interdisciplinary collaborations to policy to culture is published 6
  times per year by Fermilab and SLAC.
  \url{http://www.symmetrymagazine.org/}

  \begin{itemize}
  \tightlist
  \item
    twitter:
    \href{https://twitter.com/symmetrymag}{\texttt{@symmetrymag}}
  \end{itemize}
\end{itemize}

\subsection{Art in Physics}\label{art-in-physics}

\begin{itemize}
\item
  \textbf{Arts@CERN:} Arts at CERN promotes the dialogue between artists
  and particle physics. \url{http://arts.cern/}

  The Collide@CERN residency programme aims to develop expert knowledge
  in the arts through the connection with fundamental science. Since
  2011 the COLLIDE award calls to artists to win a fully funded
  residency for up to 3 months. \url{http://arts.cern/collide}

  Accelerate@CERN is a country specific one-month research award for
  artists who have never spent time at a science lab before.
  \url{http://arts.cern/accelerate}
\item
  \textbf{Art of Physics Competition:} The Canadian Association of
  Physicists organizes this competition, the first was launched in 1992,
  with the aim of stimulating interest, especially among non-scientists,
  in some of the captivating imagery associated with physics. The
  challenge is to capture photographically a beautiful or unusual
  physics phenomenon and explain it in less than 200 words in terms that
  everyone can understand.
  \url{https://www.cap.ca/programs/art-physics/}
\item
  \textbf{Fermilab Art Gallery:} Arts program dedicated to the
  interaction between Art and Science.
  \url{http://events.fnal.gov/art-gallery/}
\end{itemize}

\subsection{Blogs and Twitter}\label{blogs-and-twitter}

Lists of active blogs and tweets can be found on INSPIRE:

\begin{itemize}
\item
  \textbf{Scientist blogs:} \url{http://tinyurl.com/nmku27s}
\item
  \textbf{Scientists with twitter accounts:}
  \url{http://tinyurl.com/nrg5k63}
\item
  \textbf{Experiments with twitter accounts:}
  \url{http://tinyurl.com/q86kma8}
\item
  \textbf{Institutions with twitter accounts:}
  \url{http://tinyurl.com/mzcm3nw}
\end{itemize}

List of physicists on Twitter at TrueSciPhi:
\url{http://truesciphi.org/phy.html}

Some selected particle physics related blogs:

\begin{itemize}
\item
  \textbf{ATLAS blog:} \url{https://atlas.cern/updates/blog}
\item
  \textbf{CERN Love:} \url{http://www.cernlove.org/blog/}
\item
  \textbf{Life and Physics:} Jon Butterworth's blog in the Guardian.
  \url{http://www.guardian.co.uk/science/life-and-physics}
\item
  \textbf{Of Particular Significance:} Conversations about science, with
  a current focus on particle physics, with theoretical physicist Matt
  Strassler. \url{http://profmattstrassler.com/}
\item
  \textbf{Particle People:} This interactions.org page highlights a new
  blogger involved in particle physics research each month.
  \url{http://www.interactions.org/particle-people}
\item
  \textbf{Preposterous Universe:} Theoretical physicist Sean Carroll's
  blog. \url{http://www.preposterousuniverse.com/}
\item
  \textbf{Quantum diaries:} Thoughts on work and life from particle
  physicists from around the world, from 2005 to 2016.
  \url{http://www.quantumdiaries.org/}
\item
  \textbf{Quantum diaries survivor:} Experimental particle physicist
  Tommaso Dorigo's blog.
  \url{https://www.science20.com/quantum_diaries_survivor}
\item
  \textbf{Science blogs:} Launched in January 2006, ScienceBlogs
  features bloggers from a wide array of scientific disciplines,
  including physics.
  \url{http://scienceblogs.com/channel/physical-science/}
\item
  \textbf{AstroBetter:} Blog with tips and tricks for professional
  astronomers. \url{https://www.astrobetter.com/}
\end{itemize}
