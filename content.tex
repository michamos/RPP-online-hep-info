\section{Introduction}\label{databases:sec:introduction}
\index{Directories, online, people, and organizations}
Online resources are used in a diverse and expanding set of ways in particle physics. 
It is not practical to maintain a comprehensive set of links to these resources, 
particularly in print form. Given the searchability of online resources, it is also 
not particularly useful. However, some resources are central enough to our collective 
work and/or useful in ways that are not commonly known.  A curated, and of necessity 
incomplete, list is provided here. 


The collection of online information resources in particle physics and
related areas presented in this chapter is of necessity incomplete. An
expanded and regularly updated online version can be found at:
\url{http://library.cern/particle_physics_information}

Suggestions for additions and updates are very welcome.\footnote{Please
  send comments and corrections to micha.moshe.moskovic@cern.ch}

\section{Particle Data Group (PDG)
resources}\label{databases:sec:resources}

\begin{itemize}
\index{Databases!Review of Particle Physics (RPP)}
\item
  \textbf{Review of Particle Physics (RPP):} A comprehensive report on
  the fields of particle physics and related areas of cosmology and
  astrophysics, including both review articles and a
  compilation/evaluation of data on particle properties. The review
  section includes articles, tables and plots on a wide variety of
  theoretical and experimental topics of interest to particle physicists
  and astrophysicists. The particle properties section provides tables
  of published measurements as well as the Particle Data Group's best
  values and limits for particle properties such as masses, widths,
  lifetimes, and branching fractions, as well as an extensive summary of
  searches for hypothetical particles. RPP is published as a large book
  every two years, with partial updates made available once each year on
  the web.

  All the contents of the book version of RPP are available online:
  \url{http://pdg.lbl.gov}

  The printed book can be ordered:
  \url{http://pdg.lbl.gov/2019/html/receive_our_products.html}

  Of historical interest is the complete RPP collection which can be
  found online: \url{http://pdg.lbl.gov/rpp-archive/}
  \url{http://library.cern/PDG_publications/review_particle_physics}
\item
  \textbf{Particle Physics booklet:} An abridged version of the Review
  of Particle Physics, available as a pocket-sized 250-page booklet. It
  is one of the most useful summaries of physics data. The booklet
  contains an abbreviated set of reviews and the summary tables from the
  most recent edition of the Review of Particle Physics.

  The PDF file of the booklet can be downloaded:
  \url{http://pdg.lbl.gov/current/booklet.pdf}

  The printed booklet can be ordered:
  \url{http://pdg.lbl.gov/2019/html/receive_our_products.html}
\item
\index{Databases!PDGLive, PDG web browser}
\index{PDGLive, PDG web presentation}
  \textbf{PDGLive:} A web application for browsing the contents of the
  PDG database that contains the information published in the Review of
  Particle Physics. It allows one to navigate to a particle of interest,
  see a summary of the information available, and then proceed to the
  detailed information published in the Review of Particle Physics. Data
  entries are directly linked to the corresponding bibliographic
  information in INSPIRE. \url{http://pdglive.lbl.gov}
\item
\index{Databases!RPP computer-readable files}
  \textbf{Computer-readable files:} Data files that can be downloaded
  from the PDG include tables of particle masses and widths, PDG Monte
  Carlo particle numbers, and cross-section data. The files are updated
  with each new edition of the Review of Particle Physics.
  \url{http://pdg.lbl.gov/current/html/computer_read.html}
\end{itemize}

\section{Particle physics information
platform}\label{databases:sec:platforms}

  \textbf{INSPIRE:} INSPIRE serves as a one-stop information platform
  for the particle physics community, comprising interlinked databases
  on literature, authors, jobs, seminars, conferences, institutions and
  experiments (each described in more detail below). Run in collaboration by CERN, DESY,
  Fermilab, IHEP, IN2P3, and SLAC, it has been serving the scientific
  community for almost 50 years. Previously known as SPIRES, it was the
  first website outside Europe and the first database on the web. Close
  interaction with the user community and with arXiv, ADS, HEPData,
  ORCID, PDG and publishers is the backbone of INSPIRE's evolution.
  Since 2020, it is running on a modernized platform that is continuously being improved.
  \url{http://inspirehep.net/}

\section{Literature Databases}\label{databases:sec:literature}

\index{Databases!literature}
\index{Databases!literature!arXiv.org}
\index{Databases!literature!INSPIRE}
\index{Databases!INSPIRE!Literature}

Most research articles in the field of high-energy physics are first made
available on the arXiv eprint server, where researchers can learn about the
latest developments by browsing through the new announcements 
five days a week. They are indexed, together with all other publications in
high-energy physics and related fields, in the INSPIRE literature collection.
For neighboring fields such as astrophysics or mathematics, other databases are
more exhaustive.

\begin{itemize}
\item
  \textbf{ADS:} The SAO/NASA Astrophysics Data System is a Digital
  Library portal offering access to 13 million bibliographic records in
  Astronomy and Physics. The ADS search engine also indexes the
  full-text for approximately four million publications in this
  collection and tracks citations, which now amount to over 80 million
  links. The system also provides access and links to a wealth of
  external resources, including electronic articles hosted by publishers
  and arXiv, data catalogs and a variety of data products hosted by the
  astronomy archives worldwide. The ADS can be accessed at:
  \url{http://ads.harvard.edu/}
\item
  \textbf{arXiv.org:} A repository of full-text articles in physics,
  astronomy, mathematics, computer science, statistics, nonlinear
  sciences, quantitative finance, quantitative biology, electrical
  engineering and systems science, and economics. Papers are submitted
  by registered authors to arXiv, often as preprints in advance of
  submission to a journal for publication; includes postprints, working
  papers, and other relevant material. Established in 1991, the
  repository is interlinked with ADS and INSPIRE, among others. Readers
  can browse subject categories or search by author, title, abstract,
  date, and other fields. Receive daily update alerts for subfields by
  email or RSS. \url{https://arXiv.org}

\item
  \textbf{INSPIRE Literature:} The literature collection, the flagship of the INSPIRE
  suite, serves more than 1.4 million bibliographic records with a
  growing number of full-text articles attached and metadata including
  author affiliations, abstracts, references, experiments, keywords as
  well as links to arXiv, ADS, PDG, HEPData, publisher platforms and
  other servers. It provides fast metadata searches that can be easily refined using facets, plots
  extracted from full text, author disambiguation, author profile pages
  and citation analysis and is expanding its content to, e.g.,
  experimental notes. \url{http://inspirehep.net}
\item
  \textbf{MathSciNet:} This database of almost 3 million items provides
  reviews, abstracts and bibliographic information for much of the
  mathematical sciences literature. Over 100,000 new items, most of them
  classified according to the Mathematics Subject Classification, and
  more than 80,000 reviews of the current published literature are added
  each year. Author identification allows users to search for
  publications by author and citation data allows users to track the
  history and influence of research publications.
  \url{http://www.ams.org/mathscinet}
\end{itemize}

\section{Journals}\label{databases:sec:journals}

\textbf{TODO: add description and link to page}

\section{Conference and seminars databases}\label{databases:sec:conference}
\index{Databases!INSPIRE!conferences}
\index{Databases!INSPIRE!seminars}
\index{Conference databases}
\begin{itemize}
\tightlist
\item
  \textbf{INSPIRE Conferences:} The database of more than 24,000 past,
  present, and future conferences, schools, and meetings relevant to
  high-energy physics and related fields is searchable by title,
  acronym, series, date and location. Included are information about
  published proceedings, links to conference contributions in the
  INSPIRE HEP database, and links to the conference website when
  available. New conferences can be submitted from the entry page.
  \url{http://inspirehep.net/conferences}
\item
  \textbf{INSPIRE Seminars:} Created to support the surge of online seminars during
  the COVID19 pandemic, this database already contains more than 1,500
  seminars in high-energy physics and related fields. Seminars can be filtered
  by date, series and subject and exported to a calendar. Direct links to join
  the online seminar and external resources are included. All seminars are
  community-maintained and can be submitted from the entry page.
  \url{http://inspirehep.net/seminars}
\end{itemize}

\section{Research Institutions}\label{databases:sec:research}


  \textbf{INSPIRE Institutions:} INSPIRE Institutions contains over
  11,500 institutes, laboratories, and universities, where research on
  particle physics and astrophysics is led. Every record includes,
  whenever possible, as detailed information, such as address, web
  links, experiments, and links to INSPIRE papers authored by people
  affiliated to that institution. One can search for a particular
  institution by name, acronym, and location.
  \url{http://inspirehep.net/institutions}

\section{People}\label{databases:sec:people}

\begin{itemize}
\index{Databases!INSPIRE!Authors}
\item
  \textbf{INSPIRE Authors:} Searchable worldwide database of over
  125,000 active, departed, retired, and deceased people associated with
  particle physics and related fields. The affiliation history of these
  researchers, their e-mail addresses, ORCIDs, web pages, experiments
  they participated in, PhD advisor, information on their graduate
  students and links to their papers and seminars are provided, as well as a user interface to update this
  information. \url{http://inspirehep.net/authors}
\item
  \textbf{ORCID}: Registry providing persistent digital identifiers
  allowing to unambiguously identify researchers. Through integration in
  key research workflows such as manuscript and grant submission, it
  supports automated linkages between scientists and their professional
  activities ensuring that their work is recognized.
  \url{https://orcid.org}
\end{itemize}

\section{Experiments}\label{databases:sec:experiments}

\index{Experiment databases}
\index{Databases!experiments}
\index{Databases!INSPIRE!Experiments}
  \textbf{INSPIRE Experiments:} Contains more than 3,500 past, present,
  and future experiments in particle physics. It lists and classifies both accelerator and
  non-accelerator experiments. Includes official experiment name and
  number, location, and collaboration lists. Simple searches by
  participant, title, experiment number, institution, date approved,
  accelerator, or detector, return a description of the experiment,
  including a complete list of authors, title, overview of the
  experiment's goals and methods, and a link to the experiment's web
  page if available. Recently, it has expanded its scope to include
  particle accelerators besides experiments and to link them together.
  \url{http://inspirehep.net/Experiments}

\section{Jobs}\label{databases:sec:jobs}

  \textbf{INSPIRE Jobs:} Lists academic and research jobs in high
  energy physics, nuclear physics, accelerator physics and astrophysics
  with the option to post a job or to receive email notices of new job
  listings. More than 200 jobs are currently listed.
  \url{http://inspirehep.net/jobs}

\section{Software packages and
repositories}\label{databases:sec:repositories}

A vast number of software tools are used for various aspects of high-energy
physics research. Due to their number, their often specialized purpose and the
quickly-changing nature of software, no attempt has been made to present them in this chapter.
The accompanying online version \textbf{TODO: is that what we call it?}
contains an extensive categorized list of software.

\section{Data repositories and preservation}\label{databases:sec:datarepositories}

Initiatives to preserve and disseminate data produced during different stages
of research, from Monte Carlo events to machine-readable versions of tables in
papers, have grown in recent years.  Unfortunately, there currently exists no
central resource aggregating all these data as is the case for the more
established means of scholarly communication. Instead, there is a number of
repositories targeting different types of data and research subjects. They are
listed in the online version.

\index{Education databases}
\index{Collaboration databases}
\section{Particle Physics Education and Outreach
}\label{databases:sec:edusites}

