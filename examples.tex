%Examples of how to write PDG reviews


\section{Introduction to PDG review example}

This file, {\tt examples.tex}, contains examples for writing PDG
reviews. When you start writing your review, you should
comment out the line in databases-main.tex that includes this file.\\

PDG review source files consist of files edited by the review author as well as generated files. Do NOT edit generated files - your changes will be lost as the files are periodically regenerated.\\
Files edited by review authors:
\begin{itemize}
\item BASENAME-main.tex - this file contains the text of your review (you may include other files)
\item BASENAME-booklet.tex - contents of the booklet version (if there is one)
\item BASENAME-preamble.tex - for review-specific definitions or packages that need to go into the document's preamble
\item BASENAME.bib - BibTeX bibliography entries (see below)
\item figures - directory where to put all figures
\end{itemize}
Generated files (do not edit them!):
\begin{itemize}
\item Makefile - Makefile to generate different formats
\item pdg.cls - PDG review sytle file
\item pdg.bst - BibTeX style file
\item pdgdefs.tex - PDG standard symbols and macros
\item BASENAME.tex - driver file for this review in standalone mode
\item examples.tex
\end{itemize}


\section{Type-setting style}
We give here our conventions on type-setting style. Particle symbols are italic (or slanted) characters: \en, \pbar, \Lb, \pizero, \Klong, \Dstar. Charge is indicated by a superscript: $B^{-}$, $\Delta^{++}$. Charge is not normally indicated for $p$, $n$, or the quarks, and is optional for neutral isosinglets: $\eta$ or $\eta^{0}$. Antiparticles and particles are distinguished by charge for
charged leptons and mesons: $\tau^{+}$, $\kaon^{-}$ Otherwise, distinct antiparticles are indicated by a bar (overline): $\nbar_{\mu}$, \tbar, \pbar, \Kzerobar.



\section{How to include figures}

To add a figure, it is recommended to use the {\tt \string\pdgfigure} or {\tt \string\pdgwidefigure} enviroments
for a single-column or double-column wide figure in the book format, respectively. To include two images in one figure use the enviroment {\tt \string\pdgdoublefigure}.
The figures need to be in {\tt .pdf} format. Depending on your version of latex, running pdflatex may or may not convert the {\tt .eps} files into {\tt .pdf}. In case the conversion fails, the conversion can be done manually with variuos programs (ImageMagick on linux for example).
Make sure that the {\tt .pdf} figure is added into the subdirectory {\tt figures}, and that it is commited in svn or provided with your text.\\
The macros {\tt \string\pdgfigure} and {\tt \string\pdgwidefigure} take the following arguments:
\begin{verbatim}
\pdgfigure{name of the file in the figures directory}
{your caption }{ label }{option to determine the position}
{other options}
\end{verbatim}
The macro {\tt \string\pdgdoublefigure} takes the following arguments:
\begin{verbatim}
\pdgdoublefigure{name of the file1 in the figures directory}
{name of the file2 in the figures directory}
{your caption }{ label }{option to determine the position}
{other options}
\end{verbatim}

Good practice for the label is to use the following convention: {\tt databases:fig:some-meaningful-name}.\\
Examples on how to use these enviroments are shown below. The snippets of code can be directly included in databases-main.tex.
\begin{verbatim}
\pdgfigure{filename.pdf}{Figure with caption and label}
{databases:fig:ideogram}{}{}
\end{verbatim}
\pdgfigure{filename.pdf}{Figure with caption and label}{databases:fig:ideogram}{ht!}{}

\begin{verbatim}
\pdgdoublefigure{filename.pdf}{filename.pdf}
{Two figures, with caption and label, reduced in size}
{databases:fig:ideogram2}{ht!}{width=0.4\textwidth}
\end{verbatim}
\pdgdoublefigure{filename.pdf}{filename.pdf}{Two figures, with caption and label, reduced in size}{databases:fig:ideogram2}{ht!}{width=0.4\textwidth}
\begin{verbatim}
\pdgwidefigure{filename.pdf}
{Wide figure forced to be placed at the top of the page}
{databases:fig:ideogram3}{t}{}
\end{verbatim}
\pdgwidefigure{filename.pdf}{Wide figure forced to be placed at the top of the page}{databases:fig:ideogram3}{t}{}
To add a reference to the figure in the text, the following command can be used: {\tt \string\ref\{label\}}.
For example, to reference Figure \ref{databases:fig:ideogram} use the following code: {\tt \string\ref\{databases:fig:ideogram\}}.



\section{How to include tables}

To add a table it is recommended to use the {\tt \string\pdgtable} or {\tt \string\pdgwidetable} environments for single-column or double-column wide tables in the book format, respectively. 
It is recommended also to use {\tt \string\pdgtableheader} enviroment for the first line of the table.
The macros {\tt \string\pdgtable} and {\tt \string\pdgwidetable} take the following arguments:
\begin{verbatim}
\pdgtable{ dimension of the table }
{ your caption }{ label }{options}
\end{verbatim}
Good practice for the label is to use the following convention: {\tt databases:tab:some-meaningful-name}.\\
Examples on how to use these enviroments are shown below. The snippets of code can be directly included in databases-main.tex.
\begin{verbatim}
\begin{pdgtable}{c c c} 
{Table}{databases:tab:mytable}{h!}
\pdgtableheader{ Column 1 & Column 2 & Column 3}
row1  & 1  & 2\\
row2  & 1  & 2\\
row3  & 1  & 2\\
\end{pdgtable}

\end{verbatim}    
\begin{pdgtable}{c c c} 
{Table}{databases:tab:mytable}{h!}
\pdgtableheader{ Column 1 & Column 2 & Column 3}
row1  & 1  & 2\\
row2  & 1  & 2\\
row3  & 1  & 2\\
\end{pdgtable}

\begin{verbatim}
\begin{pdgtable}{|c | c | c | c|} 
{Multicolumn table}{databases:tab:mytable2}{h!}
\pdgtableheader{ \multicolumn{2}{c}{Column 1} & 
\multicolumn{2}{c}{Column 2}}
\pdgtableheader{ A & B& C & D }
row1  & 1 & 2 &3 \\
row2  & 1 & 2 &3 \\
\end{pdgtable}
\end{verbatim}    
\begin{pdgtable}{|c | c | c | c|} 
{Multicolumn table}{databases:tab:mytable2}{h!}
\pdgtableheader{ \multicolumn{2}{|c|}{Column 1} & 
\multicolumn{2}{|c|}{Column 2}}
\pdgtableheader{ A & B& C & D }
row1  & 1 & 2 &3 \\
row2  & 1 & 2 &3 \\
\end{pdgtable}

\begin{verbatim}
\begin{pdgtable}{c l}
{Table with footnotes}{databases:tab:table3}{}
One value & another\footnote{This is something to notice
\label{databases:foot:one}}\\
Two values\footref{databases:foot:one} & another \\
\end{pdgtable}
\end{verbatim}
\begin{pdgtable}{c l}
{Table with footnotes}{databases:tab:table3}{}
One value & another\footnote{This is something to notice\label{databases:foot:one}}\\
Two values\footref{databases:foot:one} & another \\
\end{pdgtable}
To add a reference to a table in the text, the following command can be used: {\tt \string\ref\{label\}}.
For example, to reference Table \ref{databases:tab:mytable2} use the following code: {\tt \string\ref\{databases:tab:mytable2\}}.


\section{Equations}

If you want to add equations, you need to use the {\tt equation} environment. A working example is:
\begin{verbatim}
\begin{equation}\label{databases:eq:equation}
N_{exp} = \sigma_{exp} \times \int L(t) dt
\end{equation}
\end{verbatim}
\begin{equation}\label{databases:eq:equation}
N_{exp} = \sigma_{exp} \times \int L(t) dt
\end{equation}

If you want to add a set of equation, please use the {\tt subequation} enviroment, together with {\tt allign}.
This will add a number for every equation in the array. A working example is:
\begin{verbatim}
\begin{subequations}
\label{databases:eq:equation1}
\begin{align}
A + B = C\\
D= \frac{E}{F}  
\end{align}
\end{subequations}
\end{verbatim}
\begin{subequations}
\label{databases:eq:equation1}
\begin{align}
A + B = C\\
D= \frac{E}{F}  
\end{align}
\end{subequations}

%If you only want a single equation number for the entire array, you can use {\tt \string\cr} command to separate equations, for example:
%\begin{verbatim}
%\begin{align}
%A + B = C \cr
%D= \frac{E}{F}  
%\end{align}
%\end{verbatim}
%\begin{align}
%A + B = C \cr
%D= \frac{E}{F}  
%\end{align}
%
You can also add text within equation with the {\tt \string\intertex} enviroment.
\begin{verbatim}
\begin{subequations}
\begin{align}
A+B = C  \\
\intertext{One can then add a comment or a reference here}
D = E 
\end{align}
\end{subequations}
\end{verbatim}
\begin{subequations}
\begin{align}
A+B = C  \\
\intertext{One can then add a comment or a reference here}
D = E 
\end{align}
\end{subequations}



\section{Labels and referencing}


If you are creating a new label, use the following convention: {\tt databases:type:some-meaningful-name }
with {\tt type} corresponding to one of the following options:
\begin{itemize}
\item {\tt fig} for figures
\item {\tt eq } for equation
\item {\tt tab} for tables
\item {\tt sec} for section, subsection etc..
\item {\tt foot} for footnotes.
\end{itemize}

Please, pay special attention when referencing sections, subsections, figures, table, equations in different reviews - use the {\tt BASENAME} associated
with the target review, not the {\tt BASENAME} of the review you're currently working on. 

%Table \ref{databases:tab:labels} lists the commands to use when referencing sections, subsections, figures, table, equations or different reviews:
%\begin{pdgwidetable}{ c | c  }{Referencing commands.}{databases:tab:labels}{ht!} 
%\pdgtableheader{ Type of reference & Command}
%reference to an object in databases review & {\tt \string\ref\{databases:type:some-meaningful-name\}} \\
%reference to an object in different review  & {\tt \string\ref\{BASENAME:type:some-meaningful-name\}}   \\
%reference a full review                     & {\tt \string\ref\{BASENAME\}                          }    \\
%\end{pdgwidetable}

To identify the {\tt BASENAME} of a review, login into \href{https://pdgprod.lbl.gov/pdgprod/PdgWorkspace/Reviews.action}{pdgWorkspace} (click to be redirected). Under \textbf{Reviews} select from the drop-down menu \textbf{all reviews}. Click on the title of the review you are interested in, and then select the \textbf{Technical details} tab. The \textbf{Basename} is the fist entry.

When including references or citations into caption, use the {\tt \string\protect} enviroment, as shown below:
\begin{verbatim}
\begin{pdgtable}{ c | c }
{Example on how to cite a paper {\protect \cite{InspireLabel}}
into a caption.}{}{ht!} 
\pdgtableheader{ Column 1 & Column 2}
A & B \\
\end{pdgtable}
\end{verbatim}
%\begin{pdgtable}{ c | c }{{Example on how to cite a paper {\protect \cite{InspireLabel}} into a caption}{}{ht!} 
%\pdgtableheader{ Column 1 & Column 2}
%A & B \\
%\end{pdgtable}



\section{Bibliography}

References are handled using BibTeX. To add a reference to your review:
\begin{itemize}
\item look up the reference in INSPIRE and download its BibTeX entry (see bottom of the \textbf{Information} tab for the article, under \textbf{Export}).
\item add the BibTeX entry to databases.bib file. Note the article tag assigned by INSPIRE - you can see it in the first line of the BibTeX entry, after {\tt \string\@article\{}.
\item cite the reference with {\tt \string\cite}, using the article tag assigned by INSPIRE.
\end{itemize}
In case the reference does not appear in INSPIRE, use the standard convention for the label: {\tt databases:meaningful-name}.
For example, to add a reference to the Review of Particle Phsyics (2018) 
add the following code to databases.bib:
\begin{verbatim}
@article{Tanabashi:2018oca,
      author         = "Tanabashi, M. and others",
      title          = "{Review of Particle Physics}",
      collaboration  = "Particle Data Group",
      journal        = "Phys. Rev.",
      volume         = "D98",
      year           = "2018",
      number         = "3",
      pages          = "030001",
      doi            = "10.1103/PhysRevD.98.030001",
      SLACcitation   = "%%CITATION = PHRVA,D98,030001;%%"
 }
\end{verbatim}
and then use the following snippet of code to add a reference to it in databases-main.tex:
\begin{verbatim}
\cite{Tanabashi:2018oca}
\end{verbatim}
In case you need to add multiple references within the same set of brakets, use the following code:
\begin{verbatim}
\cite{paper1,paper2}
\end{verbatim}
In case you want to cluster into one reference multiple papers, use the following code:
\begin{verbatim}
\cite{paper1,*paper2,*paper3}
\end{verbatim}
Note the use of the asterisk to signal trailing papers. 
If a paper is preceded by the asterisk, it can't be cited separately later - latex will fail and provide an error.
In general, the recommendation is to cite papers individually, without using the asterisk to group them.


\section{Booklet}

If your review has a booklet version, it needs to be prepared at the same time as you prepare your full review.
The content to be displayed in the booklet needs to be included in databases-booklet.tex. 
To test the rendering of your review in the booklet, you can run the following command:
\begin{verbatim}
make booklet
\end{verbatim}





\section{Standard PDG symbols}

The pdgdefs.tex file implements a series of useful shortcuts to typeset the rewies, such as particle symbols. 
All definitions are terminated with  \texttt{\string\xspace} , so you can simply write  \texttt{\string\ttbar} \texttt{production} instead of  \texttt{\string\ttbar \string\ production}.\\
Most Monte Carlo generators have a form with a suffix 'V' that allows you to include the version, e.g. \texttt{\string\PYTHIAV{8}} to produce \texttt{\PYTHIAV{8}}.
In case you need to define other symbols, please add them to the databases-preamble.tex file.


\begin{pdgwidetable}
   {|l l|l l|l l|} {Units}{databases:tab:def1}{}
   \showsymbol{\TeV     } &  \showsymbol{\syin} & \showsymbol{\barn     }   \\
   \showsymbol{\MeV     } &  \showsymbol{\inch} & \showsymbol{\mbarn    }   \\
   \showsymbol{\keV     } &  \showsymbol{\ft  } & \showsymbol{\microbarn}   \\
   \showsymbol{\eV      } &  \showsymbol{\km  } & \showsymbol{\nb       }   \\
   \showsymbol{\GeVc    } &  \showsymbol{\m   } & \showsymbol{\pb       }   \\
   \showsymbol{\GeVcSq  } &  \showsymbol{\cm  } & \showsymbol{\fb       }   \\
   \showsymbol{\GeVcc   } &  \showsymbol{\mm  } & \showsymbol{\invnb    }   \\
   \showsymbol{\GeVccSq } &  \showsymbol{\mum } & \showsymbol{\invpb    }   \\
   \showsymbol{\MeVc    } &  \showsymbol{\nm  } & \showsymbol{\invfb    }   \\
   \showsymbol{\MeVcc   } &  \showsymbol{\fm  } & \showsymbol{\invab    }   \\
   \showsymbol{\invps   } &  \showsymbol{\nm  } & \showsymbol{\lum      }   \\
   \showsymbol{         } &  \showsymbol{\ma  } & \showsymbol{          }   \\
   \showsymbol{\degr    } &  \showsymbol{\cma } & \showsymbol{          }   \\
   \showsymbol{         } &  \showsymbol{\mma } & \showsymbol{   	}   \\
   \showsymbol{         } &  \showsymbol{\muma} & \showsymbol{          }   \\
\end{pdgwidetable}

\begin{pdgwidetable}
   {|l l|l l|l l|} {Particles}{databases:tab:def2}{}
   \showsymbol{\pp         } &  \showsymbol{\ee           } & \showsymbol{\pizero   }   \\
   \showsymbol{\pbar       } &  \showsymbol{\epm          } & \showsymbol{\piplus   }   \\
   \showsymbol{\ppbar      } &  \showsymbol{\epem         } & \showsymbol{\piminus  }   \\
   \showsymbol{\tbar       } &  \showsymbol{\en           } & \showsymbol{\pipm     }   \\
   \showsymbol{\ttbar      } &  \showsymbol{\ep           } & \showsymbol{\pimp     }   \\
   \showsymbol{\bbar       } &  \showsymbol{\mumu         } & \showsymbol{\etaprime }   \\
   \showsymbol{\bbbar      } &  \showsymbol{\mun          } & \showsymbol{\Kzero    }   \\
   \showsymbol{\cbar       } &  \showsymbol{\mup          } & \showsymbol{\Kzerobar }   \\
   \showsymbol{\ccbar      } &  \showsymbol{\tautau       } & \showsymbol{\kaon     }   \\
   \showsymbol{\sbar       } &  \showsymbol{\taup         } & \showsymbol{\Kplus    }   \\
   \showsymbol{\ssbar      } &  \showsymbol{\taum         } & \showsymbol{\Kminus   }   \\
   \showsymbol{\ubar       } &  \showsymbol{\lepton       } & \showsymbol{\KzeroL   }   \\
   \showsymbol{\uubar      } &  \showsymbol{\leptonm      } & \showsymbol{\Kzerol   }   \\
   \showsymbol{\dbar       } &  \showsymbol{\ellm         } & \showsymbol{\Klong    }   \\
   \showsymbol{\ddbar      } &  \showsymbol{\leptonp      } & \showsymbol{\KzeroS   }   \\
   \showsymbol{\fbar       } &  \showsymbol{\ellp         } & \showsymbol{\Kzeros   }   \\
   \showsymbol{\ffbar      } &  \showsymbol{\leptonlepton } & \showsymbol{\Kshort   }   \\
   \showsymbol{\qbar       } &  \showsymbol{\ellell       } & \showsymbol{\Kstar    }   \\
   \showsymbol{\qqbar      } &  \showsymbol{\enu          } & \showsymbol{\jpsi     }   \\
   \showsymbol{\nbar       } &  \showsymbol{\munu         } & \showsymbol{\Jpsi     }   \\
   \showsymbol{\nnbar      } &  \showsymbol{\taunu        } & \showsymbol{\psip     }   \\
   \showsymbol{\neutron    } &  \showsymbol{\lnu          } & \showsymbol{\chic     }   \\
   \showsymbol{\antineutron} &  \showsymbol{\nub          } & \showsymbol{\UoneS    }   \\
   \showsymbol{\deuteron   } &  \showsymbol{\nunub        } & \showsymbol{\chib     }   \\
   \showsymbol{\Zzero      } &  \showsymbol{\nue          } & \showsymbol{\Dstar    }   \\
   \showsymbol{\Zboson     } &  \showsymbol{\nueb         } & \showsymbol{\Bd       }   \\
   \showsymbol{\Wplus      } &  \showsymbol{\nuenueb      } & \showsymbol{\Bs       }   \\
   \showsymbol{\Wminus	   } &  \showsymbol{\num          } & \showsymbol{\Bu       }   \\
   \showsymbol{\Wboson	   } &  \showsymbol{\numb         } & \showsymbol{\Bc       }   \\ 
   \showsymbol{\Wpm   	   } &  \showsymbol{\numnumb      } & \showsymbol{\Lb       }   \\
   \showsymbol{\Wmp        } &  \showsymbol{\nut          } & \showsymbol{\Bstar    }   \\
   \showsymbol{            } &  \showsymbol{\nutb         } & \showsymbol{\BoBo     }   \\
   \showsymbol{            } &	\showsymbol{\nutnutb      } & \showsymbol{\BodBod   }    \\		    
   \showsymbol{            } &	\showsymbol{              } & \showsymbol{\BosBos   }    \\		    
   \showsymbol{            } &	\showsymbol{              } & \showsymbol{\LambdaStar}  \\


\end{pdgwidetable}



\begin{pdgwidetable}
   {|l l|l l|l l|} {Hypothetical Particles}{databases:tab:def3}{}
   \showsymbol{\Azero }      &  \showsymbol{\gravino   } & \showsymbol{\slepton   }   \\
   \showsymbol{\hzero }      &  \showsymbol{\Zprime    } & \showsymbol{\sleptonL  }   \\
   \showsymbol{\Hzero }      &  \showsymbol{\Zstar     } & \showsymbol{\sleptonR  }   \\
   \showsymbol{\Hboson}      &  \showsymbol{\squark    } & \showsymbol{\sel       }   \\
   \showsymbol{\Hplus }      &  \showsymbol{\squarkL   } & \showsymbol{\selL      }   \\
   \showsymbol{\Hminus}      &  \showsymbol{\squarkR   } & \showsymbol{\selR      }   \\
   \showsymbol{\Hpm   }      &  \showsymbol{\gluino    } & \showsymbol{\smu       }   \\
   \showsymbol{\Hmp   }      &  \showsymbol{\stop      } & \showsymbol{\smuL      }   \\
   \showsymbol{\ggino }      &  \showsymbol{\stopone   } & \showsymbol{\smuR      }   \\
   \showsymbol{\chinop}      &  \showsymbol{\stoptwo   } & \showsymbol{\stau      }   \\
   \showsymbol{\chinom}      &  \showsymbol{\stopL     } & \showsymbol{\stauL     }   \\
   \showsymbol{\chinopm}     &  \showsymbol{\stopR     } & \showsymbol{\stauR     }   \\
   \showsymbol{\chinomp}     &  \showsymbol{\sbottom   } & \showsymbol{\stauone   }   \\
   \showsymbol{\chinoonep}   &  \showsymbol{\sbottomone} & \showsymbol{\stautwo   }   \\
   \showsymbol{\chinoonem}   &  \showsymbol{\sbottomtwo} & \showsymbol{\snu       }   \\
   \showsymbol{\chinoonepm}  &  \showsymbol{\sbottomL  } & \showsymbol{           }   \\
   \showsymbol{\chinotwop}   &  \showsymbol{\sbottomR  } & \showsymbol{           }   \\
   \showsymbol{\chinotwom}   &  \showsymbol{           } & \showsymbol{           }   \\
   \showsymbol{\chinotwopm}  &  \showsymbol{           } & \showsymbol{           }   \\
   \showsymbol{\nino}        &  \showsymbol{           } & \showsymbol{           }   \\
   \showsymbol{\ninoone}     &  \showsymbol{           } & \showsymbol{           }   \\
   \showsymbol{\ninotwo}     &  \showsymbol{           } & \showsymbol{           }   \\
   \showsymbol{\ninothree}   &  \showsymbol{           } & \showsymbol{           }   \\
   \showsymbol{\ninofour}    &  \showsymbol{           } & \showsymbol{           }   \\

\end{pdgwidetable}


\begin{pdgwidetable}
   {|l l|l l|} {Useful symbols for proton-proton physics}{databases:tab:def4}{}
   \showsymbol{\pT  }      &  \showsymbol{\mh}  \\
   \showsymbol{\pt  }      &  \showsymbol{\mW} \\
   \showsymbol{\ET  }      &  \showsymbol{\mZ} \\
   \showsymbol{\eT  }      &  \showsymbol{\mH} \\
   \showsymbol{\et  }      &  \showsymbol{   } \\
   \showsymbol{\HT  }      &  \showsymbol{   } \\
   \showsymbol{\pTsq}      &  \showsymbol{   } \\
   \showsymbol{\MET }      &  \showsymbol{   } \\
   \showsymbol{\met }      &  \showsymbol{   } \\
   \showsymbol{\Ecm }      &  \showsymbol{   } \\
   \showsymbol{\rts }      &  \showsymbol{   } \\
   \showsymbol{\sqs }      &  \showsymbol{   } \\

\end{pdgwidetable}
    		   
			
		   
			


\begin{pdgwidetable}
   {|l l|l l|l l|} {Monte Carlo Generators}{databases:tab:def5}{}
   \showsymbol{\ACERMC    }      &  \showsymbol{\MCatNLO   } & \showsymbol{\Comphep    }   \\
   \showsymbol{\ALPGEN    }      &  \showsymbol{\AMCatNLO  } & \showsymbol{\Prospino   }   \\
   \showsymbol{\GEANT     }      &  \showsymbol{\MCFM      } & \showsymbol{\LO         }   \\
   \showsymbol{\Herwigpp  }      &  \showsymbol{\METOP     } & \showsymbol{\NLO        }   \\
   \showsymbol{\HERWIGpp  }      &  \showsymbol{\POWHEG    } & \showsymbol{\NLL        }   \\
   \showsymbol{\Herwig    }      &  \showsymbol{\POWHEGBOX } & \showsymbol{\NNLO       }   \\
   \showsymbol{\HERWIG    }      &  \showsymbol{\POWPYTHIA } & \showsymbol{\muF        }   \\
   \showsymbol{\JIMMY     }      &  \showsymbol{\PROTOS    } & \showsymbol{\muR        }   \\
   \showsymbol{\MADSPIN   }      &  \showsymbol{\PYTHIA    } & \showsymbol{            }   \\
   \showsymbol{\MADGRAPH  }      &  \showsymbol{\SHERPA    } & \showsymbol{            }   \\
   \showsymbol{\MGMCatNLO }      &  \showsymbol{           } & \showsymbol{            }   \\
						
\end{pdgwidetable}

